\documentclass[12pt,a4paper]{book}
\usepackage{lmodern}
\usepackage{amssymb,amsmath}
\usepackage{ifxetex,ifluatex}
\usepackage{fixltx2e} % provides \textsubscript
\ifnum 0\ifxetex 1\fi\ifluatex 1\fi=0 % if pdftex
  \usepackage[T1]{fontenc}
  \usepackage[utf8]{inputenc}
\else % if luatex or xelatex
  \ifxetex
    \usepackage{mathspec}
  \else
    \usepackage{fontspec}
  \fi
  \defaultfontfeatures{Ligatures=TeX,Scale=MatchLowercase}
\fi
% use upquote if available, for straight quotes in verbatim environments
\IfFileExists{upquote.sty}{\usepackage{upquote}}{}
% use microtype if available
\IfFileExists{microtype.sty}{%
\usepackage{microtype}
\UseMicrotypeSet[protrusion]{basicmath} % disable protrusion for tt fonts
}{}
\usepackage[margin=2cm]{geometry}
\usepackage{hyperref}
\PassOptionsToPackage{usenames,dvipsnames}{color} % color is loaded by hyperref
\hypersetup{unicode=true,
            pdftitle={Notes on a design of a simple spatial sampling method (S3M) for assessing coverage of health and nutrition programmes in Liberia},
            pdfauthor={Valid International},
            colorlinks=true,
            linkcolor=Maroon,
            citecolor=Blue,
            urlcolor=Blue,
            breaklinks=true}
\urlstyle{same}  % don't use monospace font for urls
\usepackage{natbib}
\bibliographystyle{apalike}
\usepackage{longtable,booktabs}
\usepackage{graphicx,grffile}
\makeatletter
\def\maxwidth{\ifdim\Gin@nat@width>\linewidth\linewidth\else\Gin@nat@width\fi}
\def\maxheight{\ifdim\Gin@nat@height>\textheight\textheight\else\Gin@nat@height\fi}
\makeatother
% Scale images if necessary, so that they will not overflow the page
% margins by default, and it is still possible to overwrite the defaults
% using explicit options in \includegraphics[width, height, ...]{}
\setkeys{Gin}{width=\maxwidth,height=\maxheight,keepaspectratio}
\IfFileExists{parskip.sty}{%
\usepackage{parskip}
}{% else
\setlength{\parindent}{0pt}
\setlength{\parskip}{6pt plus 2pt minus 1pt}
}
\setlength{\emergencystretch}{3em}  % prevent overfull lines
\providecommand{\tightlist}{%
  \setlength{\itemsep}{0pt}\setlength{\parskip}{0pt}}
\setcounter{secnumdepth}{5}
% Redefines (sub)paragraphs to behave more like sections
\ifx\paragraph\undefined\else
\let\oldparagraph\paragraph
\renewcommand{\paragraph}[1]{\oldparagraph{#1}\mbox{}}
\fi
\ifx\subparagraph\undefined\else
\let\oldsubparagraph\subparagraph
\renewcommand{\subparagraph}[1]{\oldsubparagraph{#1}\mbox{}}
\fi

%%% Use protect on footnotes to avoid problems with footnotes in titles
\let\rmarkdownfootnote\footnote%
\def\footnote{\protect\rmarkdownfootnote}

%%% Change title format to be more compact
\usepackage{titling}

% Create subtitle command for use in maketitle
\newcommand{\subtitle}[1]{
  \posttitle{
    \begin{center}\large#1\end{center}
    }
}

\setlength{\droptitle}{-2em}
  \title{Notes on a design of a simple spatial sampling method (S3M) for
assessing coverage of health and nutrition programmes in Liberia}
  \pretitle{\vspace{\droptitle}\centering\huge}
  \posttitle{\par}
  \author{Valid International}
  \preauthor{\centering\large\emph}
  \postauthor{\par}
  \predate{\centering\large\emph}
  \postdate{\par}
  \date{2018-06-21}

\usepackage{booktabs}
\usepackage{color}
\usepackage{tcolorbox}
\usepackage{float}
\usepackage{setspace}

\onehalfspacing

\graphicspath{ {icons/} }

\newenvironment{rmdremind}
  {\begin{tcolorbox}[width=\textwidth, 
                     colback = {white}, 
                     title = {\textbf{Remember}}, 
                     colbacktitle = lightgray,
                     coltitle = black]
  \begin{includegraphics}[scale = 1]{remind.png}
  \begin{itemize}}
  {\end{itemize}
  \end{includegraphics}
  \end{tcolorbox}}

\newenvironment{rmdnote}
  {\begin{tcolorbox}[width=\textwidth, 
                     colback = {white}, 
                     title = {\textbf{Note}}, 
                     colbacktitle = lightgray,
                     coltitle = black]
  \begin{includegraphics}[scale = 1]{pencil.png}}
  {\end{includegraphics}
  \end{tcolorbox}}
  
\newenvironment{rmdexercise}
  {\begin{tcolorbox}[width=\textwidth, 
                     colback = {white}, 
                     title = {\textbf{Exercise}}, 
                     colbacktitle = lightgray,
                     coltitle = black]
  \begin{includegraphics}[scale = 1]{exercise.png}}
  {\end{includegraphics}
  \end{tcolorbox}}
  
\newenvironment{rmdbox}
  {\begin{tcolorbox}[width=\textwidth, 
                     colback = {white}, 
                     title = {\textbf{Exercise}}, 
                     colbacktitle = lightgray,
                     coltitle = black]
  \begin{includegraphics}[scale = 1]{pencil.png}}
  {\end{includegraphics}
  \end{tcolorbox}}
  
\newenvironment{rmdinfo}
  {\begin{tcolorbox}[width=\textwidth, 
                     colback = {white}, 
                     title = {\textbf{Info}}, 
                     colbacktitle = lightgray,
                     coltitle = black]
  \begin{includegraphics}[scale = 1]{info.png}}
  {\end{includegraphics}
  \end{tcolorbox}}  
  
\newenvironment{rmdwarning}
  {\begin{tcolorbox}[width=\textwidth, 
                     colback = {white}, 
                     title = {\textbf{Warning}}, 
                     colbacktitle = lightgray,
                     coltitle = black]
  \begin{includegraphics}[scale = 1]{warning.png}}
  {\end{includegraphics}
  \end{tcolorbox}}

\newenvironment{rmddownload}
  {\begin{tcolorbox}[width=\textwidth, 
                     colback = {white}, 
                     title = {\textbf{Download}}, 
                     colbacktitle = lightgray,
                     coltitle = black]
  \begin{includegraphics}[scale = 1]{download.png}}
  {\end{includegraphics}
  \end{tcolorbox}}

\usepackage{amsthm}
\newtheorem{theorem}{Theorem}[chapter]
\newtheorem{lemma}{Lemma}[chapter]
\theoremstyle{definition}
\newtheorem{definition}{Definition}[chapter]
\newtheorem{corollary}{Corollary}[chapter]
\newtheorem{proposition}{Proposition}[chapter]
\theoremstyle{definition}
\newtheorem{example}{Example}[chapter]
\theoremstyle{definition}
\newtheorem{exercise}{Exercise}[chapter]
\theoremstyle{remark}
\newtheorem*{remark}{Remark}
\newtheorem*{solution}{Solution}
\begin{document}
\maketitle

{
\hypersetup{linkcolor=black}
\setcounter{tocdepth}{1}
\tableofcontents
}
\hypertarget{simple-spatial-sampling-method-s3m}{%
\chapter*{Simple Spatial Sampling Method
(S3M)}\label{simple-spatial-sampling-method-s3m}}
\addcontentsline{toc}{chapter}{Simple Spatial Sampling Method (S3M)}

\includegraphics{figures/s3mlogo.png}

\hypertarget{introduction}{%
\chapter{Introduction}\label{introduction}}

The Simple Spatial Survey Method (S3M) was developed from the CSAS
coverage survey method as a response to the widespread adoption of
community management of acute malnutrition (CMAM) by ministries of
health. Large-scale programs need a large-scale survey method and S3M
was developed to meet that need.

S3M was designed to :

\begin{itemize}
\item
  Be simple enough for MoH, NGO, and UNO personnel without specialist
  statistical training to perform.
\item
  Provide a general survey method. S3M can be used to survey and map :
\item
  Need for and coverage of selective-entry programs such as CMAM and
  TSFP as well as universal programs such as EPI, GMP, GFD (general
  ration), and ``blanket'' SFP over wide areas.
\item
  Levels of indicators such as those for IYCF, WASH, and period
  prevalence / cumulative prevalence of ARI, fever, and diarrhoea over
  wide areas.
\end{itemize}

This document concentrates on using S3M to assess the need for and
coverage of a variety of selective-entry feeding programs. The
indicators discussed in this manual are:

\begin{itemize}
\item
  Therapeutic feeding (OTP and TSFP) programs :
\item
  Prevalence of SAM and coverage of treatment of SAM in children aged
  between 6 and 59 months.
\item
  Prevalence of MAM and coverage of treatment of MAM in children aged
  between 6 and 59 months.
\item
  Prevalence of MAM and treatment of MAM and in pregnant and lactating
  women (PLWs).
\item
  Food-based prevention of malnutrition (FBPM) programs :
\item
  Prevalence of need for and coverage of food-based prevention of
  malnutrition in younger children at risk of developing MAM and SAM.
\item
  Prevalence of need for and coverage of food-based prevention of
  malnutrition in pregnant and lactating women (PLWs) at risk of
  developing MAM and SAM.
\item
  Coverage of screening for all of the above programs.
\item
  Coverage of Behaviour Change Communication (BCC) programs focussing on
  maternal and child health and nutrition to all principal carers of
  children (usually their mothers) and all PLWs.
\end{itemize}

\hypertarget{sample}{%
\chapter{The survey sample}\label{sample}}

The survey method described here uses a two-stage sample:

\begin{itemize}
\item
  \textbf{First-stage:} We take an even (or near-even) spatial sample of
  communities from all of the communities in the survey area.
\item
  \textbf{Second-stage:} We take a sample of eligible individuals from
  each of the communities identified in the first stage of sampling.
\end{itemize}

Two-stage sampling is used in many survey methods. A typical example of
a survey method that uses a two- stage sample is the SMART method that
is commonly used for nutritional anthropometry surveys.

The main difference between the sample taken in S3M based surveys and in
SMART type surveys is that S3M based samples used a spatial sample in
the first stage whereas SMART type surveys use a proportional to
population size (PPS) sample.

The advantages of using a spatial first stage sample is that such a
sample allows us to identify where (and why) coverage is good, and where
(and why) coverage is poor. This information is essential to improving
program coverage and ensuring equitable access to services.

A spatial sample can be used to produce equivalent results to a
traditional proportional to population size (PPS) sample as is used in
(e.g.) SMART type surveys using a weighted analysis. This means that a
spatial sample can be made to act as a PPS sample. A PPS type sample
cannot, however, be made to act as a spatial sample.

\hypertarget{stage1}{%
\chapter{The first stage sample}\label{stage1}}

\hypertarget{step-1-find-a-map}{%
\section{Step 1: Find a map}\label{step-1-find-a-map}}

The first step in a S3M survey is to find a map of the survey area. A
map showing the locations of all towns and villages in the survey area
is essential. Try to find a map showing the locations of all towns and
villages in the survey area. You may need to update the map to take into
account migration and displacement.

For the coverage survey of 2 counties in Liberia, it will be practical
and useful to have:

\begin{itemize}
\tightlist
\item
  A small scale-map (a wide area map but with poor detail) of the entire
  survey area for each of the 2 counties. If the counties are contiguous
  (i.e., share borders with each other), the small scale map can be of
  the two counties together. This map does not need to show the location
  of all towns and villages in the survey area but it gives a general
  idea of where the 2 counties are located and main towns and locations
  and roads. Figure \ref{fig:smallScaleMap} is a small scale map of
  Liberia showing counties, roads and main towns and locations. Figure
  \ref{fig:smallScaleMapCounty} is a small scale map of two counties
  showing all the districts within the county, roads and main towns and
  locations.
\end{itemize}

\newpage

\begin{figure}[H]

{\centering \includegraphics{figures/smallScaleMap-1} 

}

\caption{Small scale map of Liberia showing counties, roads and points of interest}\label{fig:smallScaleMap}
\end{figure}\newpage

\begin{figure}[H]

{\centering \includegraphics{figures/smallScaleMapCounty-1} 

}

\caption{Small scale map of two counties in Liberia showing all districts, roads and points of interest}\label{fig:smallScaleMapCounty}
\end{figure}\newpage

\begin{itemize}
\tightlist
\item
  A collection of larger scale maps (a small area map but with good
  detail) of each of the selected counties and each of the districts
  within those counties in Liberia. Figure \ref{fig:largeScaleMapCounty}
  is a large scale map of Montserrado county showing all districts,
  roads and all settlements. Figure \ref{fig:largeScaleMapDistricts} is
  a collection of large scale maps of each of the districts of
  Montserrado country showing all roads and all settlements.
\end{itemize}

~

\begin{figure}[H]

{\centering \includegraphics{figures/largeScaleMapCounty-1} 

}

\caption{Large scale map of Montserrado county in Liberia showing all districts, roads and all settlements (towns, villages)}\label{fig:largeScaleMapCounty}
\end{figure}

\newpage

\begin{figure}[H]

{\centering \includegraphics{figures/largeScaleMapDistricts-1} 

}

\caption{Large scale maps of 5 districts of Montserrado county in Liberia showing roads and all settlements (towns, villages)}\label{fig:largeScaleMapDistricts}
\end{figure}

\newpage

The small-scale maps in Figures \ref{fig:smallScaleMap} and
\ref{fig:smallScaleMapCounty} will be useful for identifying initial
sampling locations.

The large-scale maps in Figures \ref{fig:largeScaleMapCounty} and
\ref{fig:largeScaleMapDistricts} will be useful for identifying the
precise location of sampling points and for selecting the communities to
be sampled.

\newpage

\hypertarget{step-2-decide-the-area-to-be-represented-by-each-sampling-point}{%
\section{Step 2: Decide the area to be represented by each sampling
point}\label{step-2-decide-the-area-to-be-represented-by-each-sampling-point}}

The easiest way of thinking about this is as a function of the intended
maximum distance (\(d\)) of any community from the nearest sampling
point (see Figure \ref{fig:distance}.

~

\begin{figure}[H]

{\centering \includegraphics[width=16.67in]{figures/step2} 

}

\caption{Conceptual presentation of the area represented by each sampling point}\label{fig:distance}
\end{figure}

~

There are other ways of thinking about \(d\). These are:

\begin{itemize}
\tightlist
\item
  \textbf{The area of each triangular tile}: This can be calculated
  using the formula:
\end{itemize}

\[ A ~ = ~ \tan30^ \circ ~ \times ~ \frac{9}{4} ~ d ^ 2 \]

For \(d ~ = ~ 10 ~ \text{km}\) the area of each triangular tile will be
about:

\[ A ~ = ~ \tan30^ \circ ~ \times ~ \frac{9}{4} ~ d ^ 2 ~ \approx ~ 1.3 ~ \times ~ 100 ~ = ~ 130 ~ \text{km} ^ 2 \]

\begin{itemize}
\tightlist
\item
  \textbf{Practicability}: Most of the time spent in the field when
  doing a survey will be in travelling to and from sampling points.
  Having many sampling points can make for an expensive and / or lengthy
  survey. If you know how many sampling points that you can afford to
  take (\(m\)) then you can make a \textbf{very approximate} estimate of
  a suitable value for d using the following \emph{rule-of-thumb}
  formula:
\end{itemize}

\[ d ~ \approx ~ \sqrt{\frac{\text{Program Area}}{m}} \]

The value of d calculated using this formula is approximate and should
be used as a starting point for a number of trial samples using the
procedure outlined below.

S3M surveys have been done using a wide range (i.e.~from
\(d ~ = ~ 8 ~ \text{km}\) to \(d ~ = ~ 33 ~ \text{km}\)) of values for
\(d\). A value for \(d\) of \(10 ~ \text{km}\) or \(12 ~ \text{km}\)
will probably be small enough in most circumstances.

\hypertarget{stage2}{%
\chapter{The second stage sample}\label{stage2}}

\hypertarget{analysis}{%
\chapter{Analysis}\label{analysis}}

\bibliography{book.bib}


\end{document}
